\documentclass[12pt]{article}
\usepackage[utf8x]{inputenc}
\usepackage[T1]{fontenc} 
\usepackage[ngerman, english]{babel}
\usepackage{fixltx2e}
\usepackage{hyperref}
\usepackage{amsmath}
\usepackage{graphicx}
\usepackage{caption}
\usepackage{subcaption}
\usepackage{float}
%\usepackage{adjustbox}
%\usepackage{subscript}
\author{Konstantin Murasov} 
\title{Simulationen von Memristivität und Wasserstoff-Detektion in Titandioxid.}
\begin{document}

\begin{center}

Studiengang "Medizinische Physik" der Heinrich-Heine-Universität Düsseldorf\\

\vspace{5cm}

\large Production and Characterization of CdSe Quantum Dots and their Incorporation into Porous Titaniumdioxide Structures.\\

%\normalsize Numerische Simulation eines Proton-Imaging Experiments
\vspace{5cm}



%Vorgelegt von:\\
%Konstantin Murasov\\
%Wangeroogestr 2\\
%40468 Düsseldorf\\
%konstantin.murasov@hhu.de\\
%Matrikelnummer: 2032393\\
\end{center}
\begin{figure*}[b]
Abgabedatum : 29 Juni 2016\\
%\\
%Erstgutachter: %Dr. Götz Lehmann
%\\
%Zweitgutachter:%Prof. Dr. Dr. Carsten Müller
\end{figure*}

%\newpage
%.
%\newpage
%\large Eidesstattliche Erklärung zur Bachelorarbeit\\
%\normalsize
%\\
%Ich versichere, die Bachelorarbeit selbstständig und lediglich unter Benutzung der angegebenen Quellen und Hilfsmittel verfasst zu haben.
%\\
%Ich erkläre weiterhin, dass die vorliegende Arbeit noch nicht im Rahmen eines anderen Prüfungsverfahrens eingereicht wurde.
%\\
%\\
%Düsseldorf den 29.6.16 %_____________________________________\)
%
	%\newpage
	%.
	\newpage
	\tableofcontents
	\newpage

%\begin{flushright}
%\small	\textit{Vieles von dem, was wir ohne nachzudenken tun, wird erst dann kompliziert, wenn wir es auf intellektuelle Weise betrachten. Es ist möglich, soviel über eine Sache zu wissen, daß man völlig unwissend wird.\\
%- Frank Herbert, Die Ordensburg des Wüstenplaneten}
%\end{flushright}

\normalsize	
		\section{Memristor}
		\label{sec:Memristor}
	In 1971 Leon Chua proposed a new passive circuit element, which changes its properties depending upon the cumulative amount of charge flowing through it - the memristor\cite{Missing}. The desired properties of a memristor were later found in nanoscale systems, which exhibited doping ion migration \cite{Missing}. Furthermore Strukov et al.\cite{Missing} state that "no combination of nonlinear resistive, capacitive and inductive components can duplicate the circuit properties of a nonlinear memristor". They also provide a simple (mathematical)
model for resistance switching in a memristive circuit element :

For a given sample of the length D, with a doped region of low resistivity \(R_{ON}\) of the width $\omega$ and an undoped region of high resistivity \(R_{OFF}\) of the width \(D-\omega\) the total resistivity is : 

\begin{equation}
R(W) = R_{ON}\cdot\omega+R_{OFF}\cdot(D-\omega)
\end{equation} 
Assuming an average doping ion mobility $\mu$ leads to :
\begin{equation}
\frac{d\omega (t)}{dt} = \mu \frac{R_{ON}}{D}i(t),
\end{equation}
and thus :
\begin{equation}
\omega (t) = \mu \frac{R_{ON}}{D}q(t).
\end{equation}

\begin{figure}[htb]
	\centering
		\includegraphics{C:/Users/KM/Documents/Uni/Masterarbeit/LATEX/Images/Doped.png}
	\label{fig:Doped}
	\caption{The model of resistive sitching, as proposed in \cite{Missing}: equivalent circle (left) and scheme of a sample (right). $\omega$ is the state variable, reflecting the fraction of device length,which experiences a increased concentration of doping ions.}
\end{figure}


%"Important applications include unltradense, semi-non-volatile memories and lerning networks that require a synapse-like function"\cite{missing}.

\section{Titaniumdioxide}
	Titaniumdioxide is a metal-oxide with semiconductor-like properties, which is susceptible to electroforming due to migration of Oxygen vacancies, which act as donors \cite{Effect, Strungaru}. Due to this effect and its wide availability, Titaniumdioxide invokes a high interest in various research fields such as nanoelectronics, photovoltaics and gas sensing \cite{Strungaru, Dynamics}.
	In its stoichometric form $TiO_2$ possesses insulating properties. However, introduction of Oxygen vacancies improves the conductance, since they act as donors \cite{Strungaru}. In sufficient high electric fields, the vacancies are filled by neighboring Oxygen atoms, creating the impression of a moving, +2-charged vacancy \cite{Missing}. The Titaniumdioxide containing the vacancies is denoted as $TiO_{2-\delta}$. Though the root idea is clear, the whole mechanism is still not well understood, because the examined devices also express non-linear electronic properties, which make the effects of electroforming less apparent \cite{Family}.
	
	The samples used in our workgroup are prepared via plasma electrolytic oxidation. In this procedure a Titanium foil of 0.4 mm thickness is submerged in sulfuric acid solution and set under voltages up to 200 V \cite{Titan1}. Under this conditions the oxide passivization layer on top of Titanium undergoes a electrochemical transformation: since the passivization layer is a insulator, the current seeks the path of least resistance through microscopic weak-spots in the oxide layer, created by the acid. When such a weak-spot emerges, it is forced to support a current of $\approx$ 200 mA.
	The direct result of the large current density is the splitting of water molecules into Oxygen and Hydrogen ions, which instantly achieve plasma state due to the same current that produced them. The plasma reaches temperatures up to 40000 K \cite{Titan2} and heats the metal, while simultaneously evaporating the acid solution. The escaping water vapor shapes the Titanium, while the Oxygen ions from the plasma deposit themselves on top of it, sealing it against the acid and raising the electrical resistance at the site of the discharge \cite{Titan2}.
	The result is the formation of microscopic cavities, which consist of Titaniumdioxide. Growing on top of each other, they create a porous network on top of the metal. It is reported, that higher voltages lead to bigger pores, while the depth of the $TiO_2$ layer correlates with the duration of voltage application. In \cite{Dynamics} the reported depth value is 5 $\mu$m, which has been used in following simulations.
	
	%The  implementation of Titaniumdioxide as a circuit element is based on its hysteretic behaviour $d\vec{ue}$ to migration of Oxygen defects, which act as donors [Quelle]. The gas 

\subsection{Electroforming}
	The term electroforming describes a process, when an application of high voltage or current changes the conductivity of the material\cite{Effect}. In Titaniumdioxide this occurs due to migration of Oxygen defects. If the movement of doping ions is not uniform along the cross-section of the semiconductor, formation of conductive channels inside of insulating Titaniumdioxide-matrix \cite{Effect}. It is also notable, that a strong positive voltage is able to ionize Oxygen atoms, setting it free and thus creating additional vacancies \cite{Effect, }Quelle ??.
	HYSTERETIC
	
	\begin{figure}[htb]
		\centering
			\includegraphics{C:/Users/KM/Documents/Uni/Masterarbeit/LATEX/Images/Drift.png}
		\caption{Titanoxid}
		\label{fig:Drift}
	\end{figure}
	
\section{Schottky Diode}

\begin{figure}[htb]
	\centering
		\includegraphics{Images/BeforeContact2.png}
	\caption{Metal and Semiconductor before Contact}
	\label{fig:BeforeContact}
\end{figure}

	The current through a metal-semiconductor interface depends upon the energy necessary to inject charge carriers across the interface.
It can be estimated by comparing the work function of the metal $\Phi _M$ to the electron affinity $\chi$ of the semiconductor, following Mott's equation \cite{Fundamentals}:
\begin{equation}
	\Phi _{Barrier} = \Phi _M - (\Phi _S - E_g) = \Phi _M - \chi.
\end{equation}

Once in contact, the electron energy levels in vicinity of the interface are depleted, since there are available states in the metal at lower energies. Because the density of states in the metal is much higher than in the semiconductor, the Fermi level of the metal is almost unaffected. At the same time the effect in the semiconductor is more distinct: the electric field, build up by the electrons in the metal prevents electrons from occupying states near the interface. At the same time the doping ions in this region counteract the potential of interface electron charge, so that in a certain distance $x_S$ the properties of semiconductor before contact are restored, see figure \ref{fig:ElFeld}. 

\begin{figure}[hb]
	\centering
		\includegraphics[width=0.50\textwidth]{Images/ElFeld.png}
	\caption{Electric field in the vicinity of the interface.}
	\label{fig:ElFeld}
\end{figure}

Schottky approximation states that the charge concentration in either metal and semiconductor can be taken as ending abruptly at the depletion region width $x_M$ and $x_S$ respectively, if \( \sqrt{k_B T} << \sqrt{\Phi_{Barrier}} \) \cite{Fundamentals}. Taking into account that there is no electric field in the bulk of either metal and semiconductor, the Poisson equation can be solved in the following manner:
\begin{align*}
\frac{d \phi}{dx} | _{x = \infty} = 0, \ \ \
\frac{d \phi}{dx} | _{x = -\infty} = 0
\end{align*}

\begin{equation}
\frac{d^2 \Phi}{dx^2} = \frac{e}{\epsilon _0 \epsilon _S}
\begin{cases}
0		 			 & \text{$x < x_M$} \\ 
\rho _M		 & \text{$0 > x > x_M$} \\   
\rho _S    & \text{$0 < x < x_S$} \\ 
0		 			 & \text{$x > x_S$}
\end{cases}
\end{equation}
%Taking into account that there is no electric field in the bulk of metal and semiconductor gives 
%and choosing the arbitrary constant of the potential such that \( \phi (x) | _{x = \infty} = 0\) leads to the result :
\begin{equation}
\Phi (x) =
\begin{cases}
\Phi _{Metal}		 			 									& \text{$x < x_M$} \\ 
\Phi _{Metal}	- \frac{e}{2 \epsilon _0 \epsilon _M} \rho _M (2xx_M - x^2 )		 & \text{$0 > x > x_M$} \\
\Phi _{SC}\ \ \ - \frac{e}{2 \epsilon _0 \epsilon _S} \rho _S (2xx_S - x^2 )     & \text{$0 < x < x_S$} \\ 
\Phi _{SC} = \Phi _{Metal} + \Phi_{Barrier}			 			  & \text{$x > x_S$},
\end{cases}
\end{equation}
with $\rho _M$ and $\rho _S$ being the constant charge densities of metal and semiconductor\cite{Fundamentals}.

\begin{figure}[htb]
	\centering
		\includegraphics{Images/AfterContact.png}
	\caption{Metal and Semiconductor after Contact. The orange area represents the depletion layer.}
	\label{fig:AfterContact}
\end{figure}

Talking in terms of potential, the accumulation of electrons at the metal side of the interface bends the Fermi level in the depletion region away from the conduction band (see figure \ref{fig:AfterContact}), thus decreasing the amount of mobile charge carriers in this region.
The width of the depletion region can be calculated according to the formula :
\begin{align}
 \label{eq:DeplRegion}
 x_S = \sqrt{\frac{2 \epsilon _0 \epsilon _S}{e^- \rho}(U_0 + \Phi_{Barrier})}.
\end{align}

While the electron transfer from semiconductor to the metal is hindered by the accumulation of the electrons at the interface, which are responsible for this insulating depletion layer, the transfer of electrons from metal into the semiconductor depends upon the difficulty of electron injection into the semiconductor conductance band. The energy conservation states, that in thermodynamic equilibrium, the electron flow from metal to semiconductor $i_{MS}$ and the flow in the opposite direction $i _{MS}$ balance each other. $i_{SM}$ can be calculated assuming the statistical average of the electrons moving in the negative x-direction, $v_0$ \cite{Fundamentals}:
\begin{align}
i_{SM}(0) & = \frac{n v_0} {4} exp \left( - \frac{\Phi _{Barrier}}{k_B T} \right) = -i_{MS}(0) \\
v_0 & = \sqrt\frac{8k_B T}{\pi m_S^*}
\end{align}
with 

%\begin{equation}
%f_M (E_c (\hat{k})) = \frac{1}{exp(\frac{(E_c (\hat{k}) - E_{F,M})}{k_B T}) + 1}
%\end{equation}
%is the Fermi distribution in the metal, and
%\begin{equation}
%\frac{\delta}{\delta k_x} \frac{E_c(\hat{k})}{h}
%\end{equation}
%the velocity distribution, leading to average velocity in x-direction :
%\begin{equation}
%i_{MS}(0) = \int_{k_z} \int_{k_y} \int_{k_x > 0} \left( \frac{\delta}{\delta k_x} \frac{E_c(\hat{k})}{h} \right) \cdot f_M(E_c(\hat{k}))\ dk_z\ dk_y\ dk_x.
%\end{equation}

The resulting so called depletion region is crucial for the function of the Schottky diode. If the doping level is low, the current in negative bias is suppressed, while high levels of doping allow for tunneling, thus providing ohmic current-voltage relation.


If W is in the range of a few nm, the contact is ohmic.

%This process carries on, until a
%The excess electrons create a , characterized by, driven by


\section{Simulations}
	\label{sec:Simulations}

\subsection{Units.py}
	Units-class carries the physical constants, which are used by the main program, like the barrier height at the $Pt/TiO_2$ interface or the lectric filed, necessary for dopand movement. (For convenience reasons, all data in Units.py is normalized to nm and eV, which simplifies the treatment of data in the calculation ...)

\subsection{Poisson Solver}
	The first component of the simulation program is the Poisson solver, which calculates the potential arising from the doping ions in the depletion region. The Poisson equation:
	\begin{equation}
	\Delta\Phi = - \frac{\rho}{\epsilon _0\epsilon_S},
	\end{equation}
	where $\Phi$ is the potential resulting from unbalanced charge carriers, $\rho$ the density of doping ions in the depletion region and $\epsilon_s$ is the permittivity of the semiconductor. Since this is a second order differential equation, the first derivative and the offset value must be obtained from boundary conditions. Because the back contact of $TiO_2$ is grounded, we can assume the value of the potential there to be 0, while the depletion region shields the electric field, so that the slope of $\Phi$ is zero in the bulk, so that the first derivative at the back contact is also 0. \\
	This equation is solved on a discrete grid reads : ??
	
	In case, when the positive bias exceeds the barrier height, there are no unbalanced carriers, so that the resulting potential is a straight line from interface potential \( \Phi_{Bias} + \Phi_{Barrier} \) to the back contact potential.
	
	\subsection{Schottky.py}
	Schottky.py is the subprogram responsible for the creation of Schottky environment, which then can be influenced by external voltage.
	It contains the constructor, which creates a one dimensional grid and saves data like the charge density, electrical potential and the resulting electric field. With access to DopingDrift.py and ChargeRedistribution.py it is able to calculate the movement of Oxygen ions and the corresponding depletion region width. At the same time it is able to plot the contained data or transfer it to a file to save it.
% to do so

	\subsection{DopingDrift.py}
	The file DopingDrift.py handles the movement of Oxygen vacancies inside the Titaniumdioxide sample.
	
		\subsubsection{DriftRK}
		The method DriftRK caclulates the changes in doping concentration due to ruling electric field. The suffix -RK hints that the Runge-Kutta method of integration has been used. In 1D it is defined as follows (Quelle):
		\begin{align*}		
		k_1 &= f(t, x) \cdot \Delta t,																			\\
		k_2 &= f(t + \frac{\Delta t}{2}, x + \frac{k_1}{2}) \cdot \Delta t, \\
		k_3 &= f(t + \frac{\Delta t}{2}, x + \frac{k_2}{2}) \cdot \Delta t, \\
		k_4 &= f(t + \Delta t, x + k3)  \cdot \Delta t,
		\\ \\
		k_0 &= -(k_1 + 2\cdot k_2 + 2\cdot k_3 + k_4)/6
		\end{align*}
		\begin{equation}
		\label{eq:DensityDrift}
		\rho(x + k_0, t + \Delta t) = \rho(x + k_0, t) + \rho(x, t),
		\end{equation}			
		
		where $\Delta x$ is the grid unit length, $\Delta t$ is the time step and $\rho (x,t)$ is the Oxygen vacancy density in the given cross section of the Titaniumdioxide. In that case, $k_n$ is the drift distance and $\rho(x + k_0, t + \Delta t)$ is the updated vacancy density, under assumption that all vacancies at the position x move according to prevalent electric field. $f(t,x)$ is then the average speed of vacancy drift:
		\begin{equation}
		f(x,t) = E(x) \mu (O^{2+})
		\end{equation}	
		with electric field at a given position E(x) and the vacancy mobility $\mu(O^{2+})$. Assuming electric fields, that point towards the interface positive brings about a minus in the definition of $k_0$, since in that case the movement of Oxygen ions towards the interface would lead to reduction of x.\\
		
		Let's have a closer look at eq. \ref{eq:DensityDrift}. Two major improvements can be done here. The first is, that $\rho$ can be saved only at certain grid positions, which leads to losing information in case of rounding. This error can be avoided by allowing $\rho$ to be saved in 2 neighboring grid points, instead of one, depending on the decimal places of $k_0/dx$ :
		\begin{align*}
		\rho(x + k_0, t + \Delta t) &= \rho(x + k_0, t) + \rho(x, t) \times (1 - k_0\% 1)										\\
		\rho(x + k_0 + sign(k_0), t + \Delta t) &= \rho(x + k_0 + sign(k_0, t)) + \rho(x, t) \times (k_0\% 1)
		\end{align*}
		with division rest operator '\%' and sign operator 'sign'. If for example, the displacement $k_0$ happens to be 5.5, then now half of $\rho(x, t)$ will be deposited in the grid cell (x+5), while the other half will end up in (x+6) position. \\
		The second improvement is considering an non-uniform binding energy for Oxygen vacancies, with the effect that only a fraction of vacancies moves, if the prevalent field is not sufficiently large. The idea comes from source \cite{Effect}, where it is reported that while at lower voltages a number of vacancies move through Titaniumdioxide to form a conducting bridge, the majority of vacancies stays immobile, until a higher voltage is applied, which drives them away from the interface.
		Considering an average minimum electric field $E_{min}$, which is needed to set the Oxygen ions in motion leads to an Ansatz :
		\begin{equation}
		DriftAmount(x) = \left(1 - exp\left(-(\frac{E(x)}{E_{min}})^2\right)\right) \cdot \Delta t.
		\end{equation}
		This way, only \((1-\frac{1}{e})\Delta t \approx 0.63 \Delta t \) of a cells vacancy density can be moved per time step.
		However, for time steps greater than one time unit (1s in our case), the variable 'DriftAmount' can exceed unity, which is unphysical. To prevent this, 'DriftAmount' must be limited at 1.
		It is also important to note, that $\rho(x, t + \Delta t)$ and $\rho(x, t)$ must be two separate lists, in order to prevent the vacancies to be moved more than once per time step.
	
		\subsubsection{Diffusion}
		According to Fick's Law, the change in the property due to diffusion is proportional to the second spatial derivative :
		\begin{equation}
			\frac{\delta u}{\delta t} = D\frac{\delta ^2 u}{\delta x^2}.
		\end{equation}
		Transcribing this to discrete form gives :
		\begin{equation}
			\frac{u^{n+1}_{i}-u^{n}_{i}}{\Delta t} = D \left[ \frac{u^{n}_{i-1} - 2 u^{n}_{i} + u^{n}_{i+1}}{\Delta x^2} \right].
		\end{equation}
		Introduing diffusion number \( s = D \frac{\Delta t}{\Delta x^2} \) leads to :
		\begin{equation}
			\label{eq:IncompleteDiff}
			u^{n+1}_{i} = u^{n}_{i} + s \left[ u^{n}_{i-1} - 2 u^{n}_{i} + u^{n}_{i+1} \right].
		\end{equation}
		This flaw of this method is that it does not guarantee that the total amount of u is constant. In our case however, the total amount of Oxygen vacancies does not change due to diffusion. To compensate for this effect, we must complement equation \ref{eq:IncompleteDiff}. For this, we must make sure, that the amount of \( u\), which is migrating to/from neighboring cells is also added/subtracted from them.
		The scheme for this process is:
		\begin{equation}
		blabla
		\end{equation}
		
	  \subsection{ChargeRedistribution}
	 We alredy know that the potential in the semiconductor reads:
	\begin{equation}
	\Phi (x) = \Phi(x = 0) - \int^{x}_{0} \int^{x}_{0} \frac{\rho (x)}{\epsilon_0 \epsilon_S} dx^2.	
	\end{equation}
	
	The problem is, that with non-uniform doping distribution, the width of the depletion region can not be estimated using the eq. \ref{eq:DeplRegion}, so that we dont know at which length the depletion region ends and the positive doping ions are balanced by present electrons.
However, we do have three pieces of information: the potential at both end of the semiconductor, as well as the electric field at the back contact. From this values, two are necessary to solve the Poisson equation, while the third can be used to adjust the width of the depletion region. In this subprogram, the accuracy of the depletion region width is examined using the potential at the interface.\\

The method 'ReloadDeltaPhi' first makes use of the solver to calculate the interface potential under current conditions. If the difference between the calculated and the applied potential is greater than 0.1 \% of $\Phi_{Barrier}$, the width of the depletion region is adjusted:
in case it is too low, the depletion region width is increased, while it is lowered otherwise. After that the potential is calculated again, and the process is repeated until the accuracy condition has been met.
		
		%The knowledge of the potential height at the interface allows us to adjust the width of depletion region, in a way that the Poisson equation yields correct potential value at the interface.
		
\newpage
\begin{thebibliography}{}


\bibitem{Missing}
						Strukov, D. et al. : The missing memristor found.					
						{\sl Quelle ??}
\bibitem{Effect}
						Jiang H.; Xia Q.:Effect of voltage polarity and amplitude on electroforming of $TiO_2$ based memristive devices. 						
						{\sl Nanoscale, 5, 3257 (2013)}
\bibitem{Strungaru}
						Strungaru, M. et al. : Interdependence of electroforming and hydrogen incorporation in nanoporous titanium dioxide.
						{\sl Applied Physics Letters 106, 143109 (2015)}
\bibitem{Dynamics}
						Cherchez, M. et al. : Dynamics of hydrogen sensing with $Pt/TiO_2$ Schottky diodes.
						{\sl Applied Physics Letters 103, 033522 (2013)}
\bibitem{Family}
						Yang J. et al : A Family of Electronically Reconfigurable Nanodevices.
						{\sl Advanced Materials, 21 (2009)}
\bibitem{Fundamentals}
						Enderlein, R. \& Horing N.J.M. : Fundamentals of semiconductor Physics and devices.
						{\sl World Scientific (1997)}
\bibitem{Titan1}
						El Achhab, M. : A microstructual study of the structure of plasma oxidized titanium foils.
						{\sl Appl. Phys, A 116:2039-2044 (2014)}
\bibitem{Titan2}
						El Achhab, M: Struktur reiner und platin-bedeckter elektrochemisch oxidierter Titanfolien und ihre Eigenschaften bei Wasserstoffoxidation 
						{\sl Heinrich-Heine-Universität (2013)}.


\bibitem{Cells} Grätzel, M. : Photoelectrochemical cells. 						
						{\sl Nature, Vol 414 (2001)}
\bibitem{Beate} Horn, Beate : Cadmiumselenid / Cadmiumsulfid Heterostruktur-Quantenpunkte in Titandioxid.
						{\sl Heinrich-Heine-Universität (2015)}.
\bibitem{Svenja} Herbertz, Svenja : Synthese und Charakterisierung von Halbleiterquantenpunkten zur Fluoreszenzmarkierung.
						{\sl Heinrich-Heine-Universität (2015)}.						
\bibitem{Small} Reiss,Peter; Protie `re, Myriam; Li,Liang : Core/Shell Semiconductor Nanocrystals.
						{\sl Small, Wiley-VCH Verlag (2009)}.
%\bibitem{Highly} Yang, YUE : Highly stable CdSe/CdS/ZnS Flourophores in Acidic Environment: Acile preparation and modification of core/shell/shell nanocrystals.{\sl Chem. Res. Chinese Universities 26(6) (2010)}.

						{\sl Computational Materials Science, vol. 45.1 (2009)}.
\bibitem{QDs-SC} Robel, Istva ´n; Subramanian, Vaidyanathan; Kuno, Masaru; Kamat, Prashant : Quantum Dot Solar Cells. Harvesting Light Energy with CdSe Nanocrystals Molecularly Linked to Mesoscopic TiO2 Films.
						{\sl Journal of Amarican Chemical Society 128 (2006)}.
\bibitem{Optic} Yu W., Qu L.,2003, Experimental Determination of the Extinction Coefficient ofCdTe, CdSe, and CdS Nanocrystals.
						{\sl Chem. Mater. 15 (2003)}.

\bibitem{Highly}Yang, Y. et al. : Highly Stable CdSe/CdS/ZnS Flourophores in acidic Environment: Acile Preparation and Modification of Core/																	shell/shell nanocrystals. {\sl Chem. Res. Chinese Universities, 26(6) (2010)}
\bibitem{Enhanced}Li, Z. et al. : Enhanced photovoltaic performance of solar cell based on front-side illuminated CdSe/CdS double-sensitized TiO2 nanotube arrays electrode.						{\sl Journal of Materials Science, 49 (2014)}
\bibitem{Wave-Function}Ning, Z. et al. : Wave-Function Engineering of CdSe/CdS Core/Shell Quantum Dots for enhanced electro transfer to a TiO\textsubscript{2} substrate.						{\sl The Journal of Physical Chemistry C, 114 (2010)}
\bibitem{Tracking}Pernik, D. R. et al. : Trackng the Absorption and electron injection rates of CdSe Quantum Dots on TiO\textsubscript{2}: linked versus direct attachment. {\sl The Journal of Physical Chemistry C, 115 (2011)}
\bibitem{Praktikum} Blachnik, N. : Grundmodul physikalische Chemie : CdSe-Nanokristalle {\sl Universität Mainz
						Institut für Physikalische Chemie (Jahr 2014)}
\bibitem{Brus}Brus, Luis : Electronic Wave functions in Semiconductor Clusters : Experiment and Theory.
						{\sl The Journal of Physical Chemistry, Vol. 90, No. 12 (1986)}
\bibitem{Marx} Gross R., Marx A. : Festkörperphysik. {\sl de Gruyter (2014)}
\bibitem{Achhab}El Achhab, M: Struktur reiner und platin-bedeckter elektrochemisch oxidierter Titanfolien und ihre Eigenschaften bei Wasserstoffoxidation {\sl Heinrich-Heine-Universität (2013)}.
\bibitem{Surface} Oura A., Lifshits V.G., Saranin A.A., Zoltov A.V., Katayama M. : Surface Science, an introduction.
						{\sl Springer (2003)}
\bibitem{Woodley} S. Woodley, C. Catlow: Structure prediction of titania phases: Implementation of Darwinian versus Lamarckian concepts in an Evolutionary Algorithm.
\bibitem{Murray} Murray C.B., Norris D.J., Bawendi M.G. : Synthesis and Characterization of Nearly Monodisperse CdE (E=S,Se,Te) Semiconductor Nanocrystalities {\sl J. Am. Chem. Soc. 115 (1993)}

\bibitem{Dickerson} Dickerson B. D. : Organometallic Synthesis Kinetics of CdSe Quantum Dots.   
						{\sl Virginia Polytechnic Institute and State University (2005)}
\bibitem{Haug}Haug H., Stephan W.:Quantum Theory of the Optical and  Electronic Properties of Semiconductors.
						{\sl River Edge, NJ: World Scientific (2004)}
\bibitem{Rotz}Grätzel M.,  Rotzinger P. : The influence of the crystal lattice structure on the conduction band energy on oxides of Titanium. 
						{\sl Chemical Physical Letters, Volume 118, Number 59 (1985)}

 



\end{thebibliography}

\end{document}
